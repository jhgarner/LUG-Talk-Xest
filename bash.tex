\documentclass[10pt]{beamer}

\usetheme[progressbar=frametitle]{metropolis}
\usepackage{appendixnumberbeamer}
\usepackage{svg}

\usepackage{booktabs}
\usepackage[scale=2]{ccicons}

\usepackage{pgfplots}
\usepgfplotslibrary{dateplot}

\usepackage{minted}

\usepackage{xspace}
\newcommand{\themename}{\textbf{\textsc{metropolis}}\xspace}

\title{Xest: Writing a Window Manager in Haskell}
\subtitle{}
\date{\today}
\author{Jack Garner}
\institute{CSM LUG}
% \titlegraphic{\hfill\includegraphics[height=1.5cm]{logo.pdf}}

\begin{document}

\maketitle

\begin{frame}{Table of contents}
  \setbeamertemplate{section in toc}[sections numbered]
  \tableofcontents[hideallsubsections]
\end{frame}

\section{X11}
\subsection{Servers and Clients}

\begin{frame}[fragile]{Servers}
  A server draws and communicates events with clients.
  \pause
  Who is a server?
  \begin{itemize}
    \item Xorg
    \item Xephyr
    \item Xming
    \item Cygwin
    \item XQuartz
    \item Xvfb
  \end{itemize}
  \pause
\end{frame}

\begin{frame}[fragile]{Clients}
  What is a client?
  \pause
  \begin{itemize}
    \item Firefox
    \item Gnome-Terminal
    \item Termite
    \item Google Chrome
    \item Steam
    \item Borderlands 2
  \end{itemize}
  \pause
  \begin{itemize}[<+->]
    \item i3bar
    \item Polybar
    \item Cairo-dock
    \item Compton
    \item Kwin
    \item Gnome
    \item Plasma
    \item i3
  \end{itemize}
  \pause
  Almost everything is a client.
  \pause
  Server side decorations?
  \pause
  Xlib?
\end{frame}
\begin{frame}[fragile]{Wayland}
  Lower Level protocol with better security, performance, and flexibility.

  Represents the future of Linux desktops.

  \pause
  Much more complicated and relatively young.
\end{frame}

\section{Displays vs. Screens vs. Windows}

\begin{frame}[fragile]{Displays}
  \pause
  Displays are not physical displays
  \pause
  :1 is the first/default display
  \pause
  One display can have multiple screens

\end{frame}

\begin{frame}[fragile]{Screens}
  \pause
  1 screen = 1 monitor
  \begin{itemize}[<+->]
    \item Can have different color settings, dpi, resolutions, etc.
    \item Windows can never span more than one screen
    \item Windows can't be moved from one screen to another
    \item Each screen has its own graphics driver running
    \item Each screen has 1 graphics card
  \end{itemize}
  \pause
  (In the olden days)
\end{frame}
\begin{frame}[fragile]{Screens}
  Enter Xinerama
  \pause
  Multiple graphics cards (and displays) = 1 screen
  \begin{itemize}[<+->]
    \item None of the screen limitations exist
    \item Window Managers can use the Xinerama library to figure out the monitor
      configuration
    \item Every rendering action has to happen on all cards
    \item Single threaded so slower than the worst card
    \item No 3D acceleration
    \item One DPI across all monitors
  \end{itemize}
  \pause
  (Also old)
\end{frame}
\begin{frame}[fragile]{Screens}
  What's left?
  \pause
  Nvidia TwinView
  \pause
  (Absolutely proprietary)
  \pause
  Enter Randr
  \begin{itemize}[<+->]
    \item Single screen (like Xinerama)
    \item Provides tools for managing monitors in one screen
    \item Uses the Xinerama interface for communication with clients
    \item Multiple graphics cards?
  \end{itemize}
\end{frame}
\pause
\begin{frame}[fragile]{Screens}
Enter Wayland
\end{frame}

\section{windows (Little w)}

\begin{frame}{The Tree}
  Windows are organized like a tree
  \pause
  Every X session has one window by default: The root window
  \pause
  New programs find the root window (using Xlib) then create themselves as
  children
  \pause
  Any window can have children who will follow the parent and never leave its
  bounds
  \pause
  \begin{itemize}
    \item Buttons
    \item Text Boxes
    \item Etc.
  \end{itemize}
  \pause
  Its like the Internet/DOM
  \pause
  (No one uses it like that anymore although toolkits like GTK still have
  remnants of it)
\end{frame}

\begin{frame}{Callbacks}
  Clients ask to be notified on certain events. Usually only one client can
  get notified about events for one window.

  Clients can have X commands redirected to them instead of the server.

  \pause
  Any program can request these events on any other window.

  \pause
  \begin{itemize}[<+->]
    \item Mouse click
    \item Keyboard click
    \item Becoming visible
    \item Being killed
    \item Changes to child windows
  \end{itemize}
  \pause
  The root window doesn't have any notifications or redirects at bootup...
\end{frame}
\begin{frame}{Properties}
  Each window has properties (AKA a dictionary) associated with it.

  \pause
  The X11 standard makes no guarantees about what data will or will not be
  stored.

  \pause
  ICCCM (Inter-Client Communication Conventions Manual)
  \begin{itemize}
    \item Clipboard
    \item Window names
    \item Geometry
    \item Relationship to other windows
    \item Polite focus capture
  \end{itemize}
  \pause
  EWMH (Extended Window Manager Hints)
  \begin{itemize}
    \item Multiple Desktops
    \item Pager information
    \item Polite window movement
    \item Type/State of window
    \item Actions others can take on a window
    \item Pinging/killing unresponsive windows
  \end{itemize}
\end{frame}

\section{Tying it all Together}
\begin{frame}{Program Flow}
  \begin{itemize}
    \item X server starts up
    \item Xest starts up
    \item Xest asks for SubstructureNotifyMask and SubstructureRedirectMask on
      the root window
    \item Xest Writes a bunch of properties to the root window
    \item Google Chrome asks to made a child of the root window
    \item Xest hears about it. Accepts/denies the request
    \item Xest If accepted, Xest reparents the window
    \item Xest updates its internal state
    \item Xest issues move/resize/restack commands to the server
    \item Xest Modifies/reads properties on the root/top level window
  \end{itemize}
\end{frame}

\section{Haskell}
\begin{frame}{A primer}
  Python is...
  \begin{itemize}
    \item Dynamically typed
    \item Scripting language
    \item Emphasizes practicality
    \item Provides the Pythonic way
    \item Strict execution
    \item Impure
  \end{itemize}
  Haskell is...
  \begin{itemize}
    \item Statically typed
    \item Compiled language
    \item Lots of theory (and weird names)
    \item Provides a million ways to abstract something
    \item Lazy evaluation
    \item Pure
  \end{itemize}
\end{frame}

\begin{frame}[fragile]{Wrapping}
  Haskell is all about putting information in the types and forcing you to deal
  with them.
  \begin{itemize}
    \item If a value can be null
    \item If a value might contain an error
    \item If a value is nondeterministic
    \item If a value depends on some kind of state
    \item If a value may or may not exist yet
    \item If a value has additional data associated with it
  \end{itemize}
\end{frame}

\begin{frame}[fragile]{Bad?}
  Isn't that annoying (See Go err, Java nulls, etc)?
  \pause
  \begin{itemize}
    \item Use better types (NonEmpty, Don't wrap everything in Maybe, etc)
    \item Use existing abstractions
    \item Use generated code
  \end{itemize}
\end{frame}

\begin{frame}[fragile]{Abstractions}
  \section*{name}
  Foldable

  \section*{When}
  Do you want to combine everything you're wrapping into one value?

  \begin{itemize}
    \item sum
    \item concat
    \item and/or
    \item any/all
    \item max/min
    \item find
  \end{itemize}

  \section*{What}
  \begin{itemize}
    \item List
    \item Trees
    \item Maps/Sequences
    \item Maybe (aka null)
    \item Either
  \end{itemize}
\end{frame}

\end{document}

\begin{frame}[fragile]{Abstractions}
  \section*{name}
  Functor

  \section*{When}
  Do you want to modify everything being wrapped?

  \begin{itemize}
    \item map
    \item A whole bunch of cool things...
  \end{itemize}

  \section*{What}
  \begin{itemize}
    \item List
    \item Trees
    \item Maps
    \item Maybe (aka null)
    \item Either
    \item IO
  \end{itemize}
\end{frame}

\end{document}

\begin{frame}[fragile]{Code}
  Let's see some code...
\end{frame}

\begin{frame}[fragile]{References}
  Recursion Schemes: https://blog.sumtypeofway.com/an-introduction-to-recursion-schemes/

  An inclomplete guide to writing a window manager: https://jichu4n.com/posts/how-x-window-managers-work-and-how-to-write-one-part-ii/

  An awesome, incomplete guide to how X works: https://magcius.github.io/xplain/article/
\end{frame}

\end{document}

%%% Local Variables:
%%% TeX-command-extra-options: "-shell-escape"
%%% End:
